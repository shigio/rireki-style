%
% Copyright (c) 1996, 2004, 2006, 2009, 2014, 2016, 2019, 2023
% Tama Communications Corporation. All rights reserved.
%
% Redistribution and use in source and binary forms, with or without
% modification, are permitted provided that the following conditions
% are met:
% 1. Redistributions of source code must retain the above copyright
%    notice, this list of conditions and the following disclaimer.
% 2. Redistributions in binary form must reproduce the above copyright
%    notice, this list of conditions and the following disclaimer in the
%    documentation and/or other materials provided with the distribution.
%
% THIS SOFTWARE IS PROVIDED BY THE AUTHOR AND CONTRIBUTORS ``AS IS'' AND
% ANY EXPRESS OR IMPLIED WARRANTIES, INCLUDING, BUT NOT LIMITED TO, THE
% IMPLIED WARRANTIES OF MERCHANTABILITY AND FITNESS FOR A PARTICULAR PURPOSE
% ARE DISCLAIMED.  IN NO EVENT SHALL THE AUTHOR OR CONTRIBUTORS BE LIABLE
% FOR ANY DIRECT, INDIRECT, INCIDENTAL, SPECIAL, EXEMPLARY, OR CONSEQUENTIAL
% DAMAGES (INCLUDING, BUT NOT LIMITED TO, PROCUREMENT OF SUBSTITUTE GOODS
% OR SERVICES; LOSS OF USE, DATA, OR PROFITS; OR BUSINESS INTERRUPTION)
% HOWEVER CAUSED AND ON ANY THEORY OF LIABILITY, WHETHER IN CONTRACT, STRICT
% LIABILITY, OR TORT (INCLUDING NEGLIGENCE OR OTHERWISE) ARISING IN ANY WAY
% OUT OF THE USE OF THIS SOFTWARE, EVEN IF ADVISED OF THE POSSIBILITY OF
% SUCH DAMAGE.
%
%	rireki.tex	Version 3.1
%
%	URL: https://www.tamacom.com/rireki-j.html
%
%----------------------------------------------------------------------------
%
% 多摩通信社からのお知らせ
%
% 「履歴書猿人」(https://www.tamacom.com/engine/rireki2/) という履歴書作成のための
% Webアプリケーションを公開しております。TeX の知識なしにTeX の履歴書が作成でき、
% 様々なカスタマイズも可能です。こちらもご利用いただけると幸いです。
%
%----------------------------------------------------------------------------
\newif\iflualatex
% lualatex を使用するには \lualatextrue を有効にしてください
%\lualatextrue
%
% ヘッダー
%
\iflualatex
\documentclass[b5j]{ltjsarticle}
\usepackage[deluxe,nfssonly]{luatexja-preset}
\usepackage{graphicx}
\else
\documentclass[uplatex,b5j]{jsarticle}
\usepackage[dvipdfmx]{graphicx}
\fi
\usepackage{rireki}
%
% オプション
%
% 下記のオプションが利用可能です。
%
\空行挿入		% 学歴と職歴の間に空行を挿入します
%\性別欄なし		% 性別欄を削除します
%\写真欄なし		% 写真欄を削除します
\begin{document}
%
% ID情報
%
\姓{\LARGE 履歴}
\名{\LARGE 一朗}
\姓読み{りれき}
\名読み{いちろう}
\性別{男}					% 男|女
\誕生日{平成元年2月3日}
\現在日付{令和元年5月14日}
\年齢{35}
%
% 顔写真
%
% 画像ファイルにはEPS フォーマット・縦横比4:3 のものをご使用ください。
% 縦を4cm に調整し、縦横比を変更せずに印刷します。
% 次のように指定します。
% \顔写真{photo.jpg}
%
\顔写真{}
%
% 現住所
%
\現住所郵便番号{123-4567}
\現住所{◯◯市◯◯町 1--2--3}
\現住所読み{まるまるし まるまるちょう}
\現住所市外局番{0123}
\現住所電話番号{45-6789}
\現住所呼び出し{◯◯ 方}
%
% 連絡先
%
\連絡先郵便番号{}
\連絡先{\tt taro@network.or.jp}
\連絡先読み{}
\連絡先市外局番{}
\連絡先電話番号{1234-56-7890}
\連絡先呼び出し{}
%
% 学歴、職歴
%
% 学歴、職歴を年月順に列挙してください。合計20個まで記入出来ます。
% 20個を超える部分は印刷されませんので、ご注意ください。
% 印刷順は、学歴=>職歴の順になります。
%
\学歴{平成1}{4}{◯◯市立◯◯高等学校 入学}      % {年}{月}{内容}
\学歴{平成2}{3}{◯◯市立◯◯高等学校 卒業}
\学歴{平成3}{4}{◯◯大学 入学}
\学歴{平成4}{3}{◯◯大学 卒業}
\学歴{平成5}{4}{◯◯大学大学院 入学}
\学歴{平成6}{3}{◯◯大学大学院 修了}
\学歴{平成7}{4}{専門学校◯◯ 入学}
\学歴{平成8}{3}{専門学校◯◯ 卒業}
\職歴{平成9}{4}{株式会社◯◯ 入社}
\職歴{平成10}{9}{株式会社◯◯ 退職}
\職歴{平成11}{10}{株式会社◯◯ 入社}
\職歴{平成12}{10}{株式会社◯◯ 退職}
\職歴{平成13}{10}{有限会社◯◯ 入社}
\職歴{平成14}{8}{有限会社◯◯ 退職}
\学歴{平成15}{9}{◯◯◯大学 入学(海外留学)}
\学歴{平成16}{9}{◯◯◯大学 中退}
\職歴{平成17}{10}{株式会社◯◯ 入社}
\職歴{平成18}{10}{株式会社◯◯ 退職}
\職歴{平成18}{10}{現在無職}
%
% 資格
%
% 資格を取得年月順に列挙してください。9つまで記入できます。
% 9つを超える部分は印刷されませんので、ご注意ください。
%
\資格{平成1}{4}{普通自動車一種免許}            % {取得年}{取得月}{資格}
\資格{平成2}{9}{自動二輪免許}
\資格{平成3}{4}{第二種情報処理技術者}
\資格{平成4}{4}{第一種情報処理技術者}
\資格{平成5}{3}{宅地取り引き主任者}
%
% 個人情報
%
% 志望の動機と本人希望記入欄はlatex のコマンドを記述できます。
%
\志望の動機{
	\begin{tabular}{ll}
	{\gt 志望の動機} & ◯◯◯◯◯◯◯◯◯◯◯◯◯◯◯◯◯\\
	{\gt 特技} & ◯◯◯\\
	{\gt 好きな学科} & ◯◯◯\\
	{\gt アピールポイント} & ◯◯◯◯◯◯◯◯◯◯◯◯◯◯◯\\
	\end{tabular}
}
\本人希望記入欄{
	私が希望する仕事の条件は下記の通りです。
	\begin{itemize}
	\item ◯◯◯◯◯◯◯◯◯◯◯◯◯◯◯◯◯
	\item ◯◯◯◯◯◯◯◯◯◯◯◯◯◯◯◯◯
	\item ◯◯◯◯◯◯◯◯◯◯◯◯◯◯◯◯◯
	\end{itemize}
}

\サイン{Your Signature}

\end{document}
